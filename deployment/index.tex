\documentclass[10pt,a4paper]{article}

\usepackage[utf8]{inputenc}
\usepackage[english]{babel}
\usepackage[english]{isodate}
\usepackage[parfill]{parskip}
\usepackage[super]{nth}
\usepackage{listings}
\usepackage{hyperref}


\pagenumbering{Roman}
%\tableofcontents
%\newpage
%\listoffigures
%\newpage
%\listoftables
%\newpage
\pagenumbering{arabic}

\title{PEAT Deployment}

\begin{document}
\maketitle

\section{Introduction}
 


\section{Docker}
This section will outline the advised methods for installing and deploying the Docker Engine on a system.

\subsection{Installation}
The Docker Engine uses Linux-specific kernel features, because of this it can run natively on any Linux operating system however Docker can be used on non-Linux operating systems such as Windows and OSX using Boot2Docker. This section will cover the installation of Docker on Ubuntu, Fedora, Windows and OSX and provide guidance for installing Docker on other systems.

Docker version 1.4 or above is advised for the PEAT platform. 

\subsubsection{Linux}
Note that as of the current stable release (1.5) the Docker Engine does not natively support 32-bit Linux architectures, although some systems have been able to run Docker on a 32-bit architecture but this is not advised.

\newpage

\paragraph{Ubuntu}
To obtain the latest version of Docker for Ubuntu 14.04 and 12.04 call the following command from a terminal.

%\begin{lstlisting}
\centerline{\textbf{curl -sSL https://get.docker.com/ubuntu/ $\vert$ sudo sh}}
%\end{lstlisting}

Systems using Ubuntu 13.04 or 13.10 can find installation instructions here.
 
\centerline{\textbf{https://docs.docker.com/installation/ubuntulinux/}}

\paragraph{Fedora}
To obtain the latest version of Docker for Fedora 21 and over call the following commands from a terminal

\centerline{\textbf{sudo yum -y install docker}}
\centerline{\textbf{sudo yum -y update docker}}

For Fedora 20 due to a naming conflict with a system tray application you must use the following commands.

\centerline{\textbf{sudo yum -y remove docker}}
\centerline{\textbf{sudo yum -y install docker-io}}
\centerline{\textbf{sudo yum -y update docker-io}}

The Docker daemon must then be started and instructed to run at boot.

\centerline{\textbf{sudo systemctl start docker}}
\centerline{\textbf{sudo systemctl enable docker}}

\subsubsection{Windows}

The Docker Engine uses Linux-specific features, because of this the program Boot2Docker is required for installing Docker on Windows. This runs a lightweight Virtual Machine using VirtualBox on the host machine. Docker is then installed on this instead of on Windows however it can be accessed easily through the host machine.

Installation instructions for the latest version of Boot2Docker can be found here.
\centerline{\textbf{https://docs.docker.com/installation/windows/}}

\subsubsection{OSX}
As with Windows installations, Boot2Docker needs to be used with OSX in order for Docker to work on an OSX machine.
The installation instructions for installing the latest version of Boot2Docker on OSX can be found here.
\centerline{\textbf{https://docs.docker.com/installation/mac/}}

\newpage
 
\subsection{Deployment}
This section will detail how to deploy the PEAT docker platform on a host machine.

\subsubsection{Configuration}
Clone the PEAT-docker repository from the PEAT GitHub account.\\
\centerline{\textbf{git clone https://github.com/peat-platform/peat-docker.git}}


\subsubsection{Orchestration}

\end{document}